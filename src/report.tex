\documentclass{article}
\usepackage[utf8]{inputenc}

\title{315 Project Report}
\author{Patrick Rock, Ben Creighton, Juaune Burgess}
\date{September 2013}

\begin{document}

\maketitle

\section{Design Changes}
Our design changed as we grew to understand the project requirements more fully. The most dramatic change 
is in the final application. We initially planned to make a calendar style app. The app would provide suggestions
based on your planned activites. This proved to be too complex to implement with the resources available to us. 
We decided instead to make a grocery list management application. This application makes use of the database by 
allowing the user to create and track grocery lists. Each grocery list is a table in the database. Lists can 
be added and delted. Our fututre development plans if we were to continue work on the application include:
allowing lists to be edited, creating a GUI, fixing the database parser to allow persitstant storage of lists. 

\section{Difficulties and Solutions}
We had major difficulties in building the parser.
Attribute lists were not being processed correctly, and the group didn’t agree on a data-flow model.

\section{Lessons Learned}
\subsection{Patrick} Testing and planning are important parts of developing software. As a group we did not spend enough
time testing our code. Our design should have been more precise. An imprecise design makes collaboration
difficult, as different people have different ideas about what the code should be. Each week we would 
finish the code and submit withought testing. This put us very behind since each week would begin with testing
and debugging last week's code. I learned that it is important to be physically together in order to get work 
done. I thought that our work could be done through github and everyone would understand, but this was not the case.
In the next project I will make it a point to meet often and work physically with my group.

\subsection{Ben}

\subsection{Juaune}

\section{Example Session}


\end{document}
