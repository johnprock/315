\documentclass{article}
\usepackage[utf8]{inputenc}

\title{315 Project Report}
\author{Patrick Rock, Ben Creighton, JuAune Burgess}
\date{September 2013}

\begin{document}

\maketitle

\section{Design Changes}
Our design changed as we grew to understand the project requirements more fully. The most dramatic change 
is in the final application. We initially planned to make a calendar style app. The app would provide suggestions
based on your planned activites. This proved to be too complex to implement with the resources available to us. 
We decided instead to make a grocery list management application. This application makes use of the database by 
allowing the user to create and track grocery lists. Each grocery list is a table in the database. Lists can 
be added and deleted. Our fututre development plans if we were to continue work on the application include:
allowing lists to be edited, creating a GUI, fixing the database parser to allow persitstant storage of lists. 

\section{Difficulties and Solutions}
We had major difficulties in building the parser.
Attribute lists were not being processed correctly, and the group didn’t agree on a data-flow model for how the
parser's functions would return necessary data to their callers. This mis-match made debugging and difficult and
completion within the timeframe very unlikely, as we would likely have had to re-write huge portions of the parser.
Some of the parsing functions work. Commands are parsed correctly. Queries are not parsed. Integration functions
for Open and Close operations. We were forced to move on when building the final application and work directly with
the database API. 

\section{Devlog}
This is a copy of the devlog from our github repository.
\begin{verbatim}

8/30/2013
All team members are working on the design document.

9/1/2013
Additions to the design document

9/5/2013
Patrick is working on setting up data structures for the engine.

9/8/2013
Patrick is working on Select, Rename, and Update
Ben is working on Projection, Difference, and Cross Product
JuAnne: Delete(table), Delete(tuple), Insert(tuple)

9/15/2013
Patrick is working on create table in the parser.

The following functions are completed:
show
create Table
Union
delete(tuple)
delete(table)
insert(tuple)
select
rename
update
insert into
projection
set difference
cross product


9/18/2013
Work Assignments: Due Friday Night
Patrick: Select, Reaname, Update, Insert
Ben: Projection, Difference, Union, Cross Product
JuAnne: Create, Close 


9/25/2013
Project: Grocery List Application

The basic application will have three functions: 
Create List, Delete list, and Show Lists.
After this is implemented we can add functions like Edit list. 
The application will make calls directly to the dbEngine. 
This will be a terminal application.

Example:

user# ./listapp
Type 1 for Create List.
Type 2 for Delete List.
Type 3 for Show Lists.
-->



There is a root menu that asks for a basic action.
Each action opens a submenu with a dialogue
for that action. Our application should have one Database object. 
Each list should be a table in this database.
Since our parser isn't functioning, our application won't have persistant storage. 
The root menu and application layout should be completed by tomorrow night.
The application will be a class. Each action is a function within
the class. 

Ben     -- Create
Patrick -- Delete
JuAnne  -- Show
\end{verbatim}

\section{Lessons Learned}
\subsection{Patrick} Testing and planning are important parts of developing software. As a group we did not spend enough
time testing our code. Our design should have been more precise. An imprecise design makes collaboration
difficult, as different people have different ideas about what the code should be. Each week we would 
finish the code and submit withought testing. This put us very behind since each week would begin with testing
and debugging last week's code. I learned that it is important to be physically together in order to get work 
done. I thought that our work could be done through github and everyone would understand, but this was not the case.
In the next project I will make it a point to meet often and work physically with my group.

\subsection{Ben}Our failure with the parser is clearly due to bad in-group communication. This is what caused the
parser to be implemented with two ideas on how it should work and forced its integration to be more difficult than
it had to be. This could have been avoided by planning as a group how the implementation should be carried out. Also,
parts of the project were completed slower than they had to be because certain parts were implemented out of order.
Certain parts were completed first which were dependent on other parts which had not been completed. This made testing
this code impossible until all its dependencies had been implemented as well. Again, planning for this would have made
development and testing go much faster and avoided waiting. 


\subsection{JuAune} Not meeting as a group was our ultimate downfall. With less frequent meeting our group was unable to 
communicate very important ideas for our project. With littles things falling through the cracks here and there our 
project hit bottom at the parser implementation. The parser turned out to be more difficult than we had planned for and 
with and testing being tossed to the wayside to make up for lost time we lost most functinality of our parser. With this 
I learned that we will need to be more available to meet with each other and not wait until the end of the week to start 
the next phase. All in all, for the next project it would be beneficial to develope a more detailed design, better 
communication and more in depth testing.

\section{Example Session}
\begin{verbatim}
> make
g++ -g main.cpp
> ./a.out
Please choose from the options listed below: 

1. Create New List 
2. Show Lists 
3. Remove List 
4. EXIT 
1


List name:
Groceries

How many items?
2

Item Type:
1. for INTEGER
2. for VARCHAR
2

Datafield Size:

6

Item Type:
1. for INTEGER
2. for VARCHAR
2

Datafield Size:
5

Enter the 0th item: Apples

Enter the 1th item: Plums

Done.
Please choose from the options listed below: 

1. Create New List 
2. Show Lists 
3. Remove List 
4. EXIT 
2


Retrieving lists...

Table: Groceries
0Contents: Apples Plums 
Please choose from the options listed below: 

1. Create New List 
2. Show Lists 
3. Remove List 
4. EXIT 
4


>
\end{verbatim}

\section{Work Load Distribution}
Patrick: 50\%\\
Ben:     40\%\\
Juaune:  10\%\\

\end{document}
